\addcontentsline{toc}{section}{Abstract}
\section*{Abstract}
\label{sec:}

The vast majority of mobile platforms available today are non-holonomic. They only have one or two independent degrees of freedom. As a result, its manoeuvrability is limited, and it frequently requires a lot of space to perform functions like turning and parking. By increasing a vehicle's degrees of freedom and manoeuvrability, it can take various complex trajectories that non-holonomic vehicles find difficult or impossible to take.
\par
As a result, this project designs and develops a prototype mobile platform with holonomic and omnidirectional motion using castor wheels. Castors are used to move heavy loads in a variety of industries. Because castor wheels are difficult to control, we intend to introduce Direct Current Motors that will enable castor wheel control. The design will include the selection of Direct Current motors and power transmission systems based on the amount of power required to move heavy loads.
\par
The goal of this design is to give casters some form of control. This control will be accomplished by varying motor speed and direction, resulting in varying degrees of motion. This control will be accomplished remotely using hand motion control or a mobile application interface, depending on the application area. This will necessitate sophisticated software development, which will be accomplished through the use of high-level programming languages such as Python and Dart while leveraging low level capabilities to increase run time. 
\par
The goal is to have a prototype mobile platform capable of carrying a load of $40kg$ at the end of the design and fabrication process. This platform will not require manual control, but will instead be operated remotely via a mobile application software or hand motion control.
